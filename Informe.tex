\documentclass[12pt, letterpaper]{article}
\usepackage{fullpage}
\usepackage[top=2cm, bottom=4.5cm, left=2.5cm, right=2.5cm]{geometry}
\usepackage[toc,page]{appendix}
\usepackage{listings}
\usepackage{amsmath} % Advanced math typesetting
\usepackage{amssymb}
\usepackage[utf8]{inputenc} % Unicode support (Umlauts etc.)
\usepackage[spanish,es-tabla]{babel} % Change hyphenation rules
\decimalpoint
\usepackage{hyperref} % Add a link to your document
\usepackage{graphicx} % Add pictures to your document
\usepackage{listings} % Source code formatting and highlighting
\usepackage{blindtext}
\usepackage{float}
\usepackage{booktabs}
\usepackage{gensymb}
\usepackage{enumitem}
\usepackage{subcaption}
\usepackage[table,xcdraw]{xcolor}
\usepackage{booktabs}
\usepackage{tabularx,ragged2e,booktabs,caption}
\usepackage[cache=false]{minted}
\usepackage{ragged2e}
\usepackage{etoolbox}
%\usepackage[T1]{fontenc}
\usepackage{geometry}

\hypersetup{colorlinks=true,urlcolor=blue}

\definecolor{mGreen}{rgb}{0,0.6,0}
\definecolor{mGray}{rgb}{0.5,0.5,0.5}
\definecolor{mPurple}{rgb}{0.58,0,0.82}
\definecolor{backgroundColour}{rgb}{0.95,0.95,0.92}

\renewcommand{\labelitemi}{$\bullet$}

\apptocmd{\frame}{}{\justifying}{}

% Framed Box Minted
%\usepackage{tcolorbox}
%\usepackage{etoolbox}
%\BeforeBeginEnvironment{minted}{\begin{tcolorbox}}%
%\AfterEndEnvironment{minted}{\end{tcolorbox}}%

\begin{document}
	\begin{titlepage}
		\begin{center}
			\vspace*{2cm}
			
			\includegraphics[scale=0.12]{images/UPB.png}
			\vfill
			
			\Huge{\textbf{TP4}}\\[5mm]
			
			\huge{\slshape{Teleinform\'atica}}
			\vfill
			
			\line(1,0){360}\\[3mm]
			\Huge{\textbf{Sockets}}\\[1mm] 
			\line(1,0){360}\\[2cm]
			\begin{center}
				\Large{Gustavo Marin}
				\\
				\Large{Alejandro Coello}
			\end{center}
			
			\vfill
			\large{\today}
		\end{center}
	\end{titlepage}
\section{Objetivo}
\noindent Enviar un fichero entre dos computadoras utilizando un socket TCP.
\section{Requerimientos}
\noindent El servidor deberá calcular throughput (i.e., el número de kilobytes por segundo – KB/s) que recibió y el porcentaje bytes recibidos del total esperados. Para simplificar, se deberá utiliza una aproximación del tiempo entre varios recv(…) que sean mayores o iguales a 1 segundo

\section{Implementaci\'on}
\noindent En el servidor se obtiene el tamaño del fichero a recibir y se recibir\'an datos hasta alcanzar el tamaño del fichero, esto se realizara mediante un {\fontfamily{qcr}\selectfont\textbf{while}}, dentro el cual se tiene la recepci\'on de datos, la escritura en el fichero, un control del tiempo, el c\'alculo del porcentaje recibido y la impresi\'on de la velocidad de transmisi\'on y el porcentaje.
\subsection{C\'alculo del porcentaje recibido}
\noindent Para calcular el porcentaje recibido se realiza un divisi\'on entre la cantidad de bytes recibidos entre el total a recibir y se multiplica por 100
\begin{minted}{c}
per = ((total_byt * 100) / fileSize);
\end{minted}
\subsection{C\'alculo de la velocidad de transmisi\'on}
\noindent Para calcular la velocidad de transmisi\'on se inicio la cuenta de un tiempo antes de entrar al ciclo {\fontfamily{qcr}\selectfont\textbf{while}} y al acabar la recepci\'on, este tiempo se ira acumulando y cuando sea mayor a 1 segundo se realizara la divisi\'on entre los bytes recibidos hasta ese momento entre el tiempo transcurrido.
\begin{minted}{c}
speed = (byps / 1000) / total_time;
\end{minted}
\subsection{Escritura del fichero}
\noindent Para realizar la escritura del fichero se utiliza {\fontfamily{qcr}\selectfont\textbf{fwrite}}, se escribira sobre un fichero previamente creado con {\fontfamily{qcr}\selectfont\textbf{fopen}}.
\begin{minted}{C}
// Creacion del fichero
FILE* pFile = fopen(fileName, "w+");
// Escritura del fichero
fwrite(buffer, sizeof(char), byt_rcv, pFile);
\end{minted}

\clearpage
\section{Resultados}
\noindent A continuaci\'on se adjuntaran im\'agenes de la aplicaci\'on funcionando.
\\~\\
En la siguiente imagen se muestra la ejecuci\'on del servidor con todos los requerimientos funcionando, adem\'as s se puede observar el formato de introducci\'on de argumentos, nombre del fichero, puerto y tamaño del buffer.
\begin{figure}[h!]
	\centering
	\includegraphics[width=0.65\linewidth]{images/server}
	\caption{Servidor}
	\label{fig:server}
\end{figure}
\\En la siguiente imagen se muestra la ejecuci\'on del cliente, adem\'as del formato de la introducci\'on de argumentos, archivo a enviar, IP del servidor, puerto y tamaño del buffer.
\begin{figure}[h!]
	\centering
	\includegraphics[width=0.65\linewidth]{images/client}
	\caption{Cliente}
	\label{fig:client}
\end{figure}

\clearpage
%\section{Conclusiones}
%\begin{itemize}
%	\item La implementaci\'on del chat con el formato indicado fue exitosa.
%	\item Nodejs es una herramienta muy \'util para el desarrollo de m\'ultiples aplicaciones.
%	\item Los paquetes de javascript son muy \'utiles para el uso r\'adido de funciones espec\'ificas.
%\end{itemize}
\begin{appendices}
\begin{minted}[obeytabs=true,tabsize=2, breaklines, breakanywhere]{C}
#include <sys/types.h>
#include <sys/socket.h>
#include <netinet/in.h>
#include <arpa/inet.h>
#include <stdio.h>
#include <stdlib.h>
#include <string.h>
#include <unistd.h>
#include <time.h>

int main(int argc, char *argv[]){

	if (argc != 4) {
		printf("usage: %s fileName port bufferSize\n", argv[0]);
		return -1;
	}
	
	// Nombre del archivo a recibir
	char* fileName = argv[1];
	
	// Numero de puerto
	int port = atoi(argv[2]);
	
	// Tamaño del buffer
	int bufferSize = atoi(argv[3]);
	
	char* buffer = malloc(bufferSize);
	
	struct sockaddr_in stSockAddr;
	int SocketFD = socket(PF_INET, SOCK_STREAM, IPPROTO_TCP);
	if(-1 == SocketFD) {
		perror("can not create socket");
		exit(EXIT_FAILURE);
	}
	
	memset(&stSockAddr, 0, sizeof(struct sockaddr_in));
	
	stSockAddr.sin_family = AF_INET;
	stSockAddr.sin_port = htons(port);
	stSockAddr.sin_addr.s_addr = INADDR_ANY;
	if(-1 == bind(SocketFD,(const struct sockaddr *)&stSockAddr, sizeof(struct sockaddr_in))) {
		perror("error bind failed");
		close(SocketFD);
		exit(EXIT_FAILURE);
	}
	
	if(-1 == listen(SocketFD, 10)) {
		perror("error listen failed");
		close(SocketFD);
		exit(EXIT_FAILURE);
	}
	
	int ConnectFD = accept(SocketFD, NULL, NULL);
	
	if(0 > ConnectFD) {
		perror("error accept failed");
		close(SocketFD);
		exit(EXIT_FAILURE);
	}
	
	long fileSize;
	int fsize = recv(ConnectFD, &fileSize, sizeof(fileSize), 0);
	if(fsize != sizeof(fileSize)) {
		printf("Error reading file size\n");
		exit(-1);
	}
	printf("Trying to receive %ld Bytes...\n", fileSize);
	
	int totalBytesReceived = 0;
	
	FILE* pFile = fopen(fileName, "w+");
	// here ADD your code
	int per, byt_rcv = 0, total_byt = 0, byps = 0;
	float seconds, total_time, speed;
	clock_t start = clock();
		while (total_byt < fileSize) {  	
		byt_rcv = recv(ConnectFD, buffer, bufferSize, 0);    
		fwrite(buffer, sizeof(char), byt_rcv, pFile);
		clock_t end = clock();
		total_byt += byt_rcv;
		byps += byt_rcv;
		per = ((total_byt * 100) / fileSize);
		seconds = ((float)(end - start) / CLOCKS_PER_SEC);
		total_time += seconds;
		if (total_time >= 1) {
			speed = (byps / 1000) / total_time;
			printf("Elapsed time %.4f s\tVelocidad %.3f KB/s\tbytes %d\n", total_time, speed, total_byt);
			printf("Expected: %d\tRead: %d\tPorcentaje %d %%\n", bufferSize, byt_rcv, per);
			total_time = 0;
			byps = 0;
		}	
	}
	
	fclose(pFile);
	
	printf("%s written\nDONE\n", fileName);
	
	shutdown(ConnectFD, SHUT_RDWR);
	
	close(ConnectFD);
	close(SocketFD);
	
	return 0;
}
\end{minted}
\end{appendices}
\end{document}